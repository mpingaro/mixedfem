\documentclass{article}

\usepackage{graphicx}
\usepackage{xcolor}
\usepackage{colortbl}
\usepackage{fp}
\usepackage{tikz, pgfplots}
\usetikzlibrary{calc}
\usetikzlibrary{patterns}
\usetikzlibrary{intersections}
\usetikzlibrary{arrows}
\tikzset{>=latex}

% Color definition
\definecolor{green2}{RGB}{154, 205, 50}
\definecolor{green3}{RGB}{141, 182, 0}
\definecolor{blue2}{RGB}{0, 102, 255}
\definecolor{lightgreen}{RGB}{178,255,102}


\begin{document}
\scriptsize
\begin{figure}[!h]
\begin{center}
\begin{tikzpicture}
 %transpose legend
 \begin{loglogaxis}[width=0.60\textwidth, height=0.60\textwidth,
  legend style={anchor=south, at={(0.5, 1.07)}, draw=none, font=\scriptsize},
  grid=major,
  legend columns=2,
  xlabel={\small Number of Elements},
  ylabel={\small $\parallel p_{h}-p \parallel_{L^{2}} / \parallel p \parallel_{L^{2}}$},
  xmin=1, xmax=1e5,
  ymin=1e-3, ymax=1e0,
  ytick={1e-3, 1e-2, 1e-1, 1e0, 1e1, 1e2},
  yticklabels={$10^{-3}$, $10^{-2}$, $10^{-1}$, $10^{0}$, 
  $10^{1}$, $10^{2}$},
  xtick={1e0, 1e1, 1e2, 1e3, 1e4, 1e5},
  xticklabels={$10^{0}$, $10^{1}$, $10^{2}$, $10^{3}$, $10^{4}$, $10^{5}$},
 ]
 % Reference
 %\addplot[black, ultra thick] coordinates{
 %(50, 1.8248e-5)
 %(15000, 1.8248e-5)
 %}; 
 %\addlegendentry{Reference} 
 %
 \addplot[green, mark=square, very thick, mark options={solid}]
 table [x index={0}, y index={1}]
 {results/elastic_error_disp_u_peers_1b.txt};
 \addlegendentry{PEERSQ1B}
 %
 \addplot[red, mark=x, very thick, mark options={solid}]
 table [x index={0}, y index={1}]
 {results/elastic_error_disp_u_peers_2b.txt};
 \addlegendentry{PEERSQ2B}
 %
 \addplot[blue, mark=o, very thick, mark options={solid}]
 table [x index={0}, y index={1}]
 {results/elastic_error_rot_u_peers_2bm.txt};
 \addlegendentry{PEERSQ2BM}
 %
 \addplot[black, mark=square, very thick]
 table [x index={0}, y index={1}]
 {results/elastic_error_rot_u_peersq.txt};
 \addlegendentry{PEERSQ}

 \end{loglogaxis}
\end{tikzpicture}
\end{center}
\caption{Regular mesh: relative error versus the number of elements measured relative 
to the $L^{2}$}
\end{figure}

\end{document}