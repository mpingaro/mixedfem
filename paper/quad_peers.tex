\documentclass[a4paper,11pt]{article}
\usepackage[utf8]{inputenc}
\usepackage[italian]{babel}
\usepackage{graphicx}
\usepackage{subfigure}
\usepackage{amsmath}
\usepackage{bm}

\usepackage{graphicx}
\usepackage{xcolor}
\usepackage{colortbl}
\usepackage{fp}
\usepackage{tikz, pgfplots}
\usetikzlibrary{calc}
\usetikzlibrary{patterns}
\usetikzlibrary{intersections}
\usetikzlibrary{arrows}
\tikzset{>=latex}

% Color definition
\definecolor{green2}{RGB}{154, 205, 50}
\definecolor{green3}{RGB}{141, 182, 0}
\definecolor{blue2}{RGB}{0, 102, 255}
\definecolor{lightgreen}{RGB}{178,255,102}


%  Sistemazione margini
\addtolength{\hoffset}{-20pt}  % Riduco spazio bordo alto
\oddsidemargin = 10pt          % Riduco spazio dordo lati
\textwidth = 500pt             % Allargo parte scritta (righe) 
\textheight = 700pt            % Allargo parte scritta (colonne)

\title{\textbf{An extension of PEERS element for quadrilateral elasticity mixed formulation}}
\date{\today}
\author{Dr. Marco Pingaro and Prof. Boffi}

\begin{document}
\maketitle

\section{Numerical example}


\subsection{Square problem}
First example is a unit square domain with homogeneous Dirichlet boundary conditions and we the exact solution is
\begin{equation}\label{eq:exact_solution}
u_{1} = \cos (\pi x) \sin(2\pi y), \hspace{10pt} u_{2} = \sin(\pi x)\cos(\pi y).
\end{equation} 
The Lamé constant are fix to $\lambda = 123$ and $\mu=79.3$.
By imposition of the previously exact solution one obtain for the body force $f$
\begin{equation}
\begin{split}
&f_{1} = -\pi^{2} \cos(\pi x) \sin(\pi y) \left( \lambda + \mu + 2\lambda\cos(\pi y) + 
12\mu\cos(\pi y)\right), \\
&f_{2} = -\pi^{2}\sin(\pi x)\left( \lambda\cos(\pi y) + 3\mu\cos(\pi y) + 2\lambda\left(2\cos(\pi y)^{2} 
- 1\right) + 2\mu\left(2\cos(\pi y)^{2} - 1\right) \right)
\end{split}
\end{equation}
The problem is study using two type of mesh, first of all using a square mesh and before using a trapezoidal mesh.
The two different types of meshes are shown in figures \ref{fig:square_regular} and \ref{fig:square_trapezoidal}.
%
\begin{figure}[h!]
\begin{center}
\subfigure[Regular mesh \label{fig:square_regular}]
{%\documentclass{article}
%
%\usepackage{tikz}
%\usepackage{tikz-3dplot}
%\usetikzlibrary{calc}
%\usetikzlibrary{patterns}
%\usetikzlibrary{intersections}
%\usetikzlibrary{arrows}
%\tikzset{>=latex}
%
%\begin{document}
%
%\begin{figure}[!h]
%\begin{center}

\begin{tikzpicture}[scale = 5.]
	\coordinate[](a) at (0,0);
	\coordinate[](b) at (1,0);	
	\coordinate[](c) at (1,1);
	\coordinate[](d) at (0,1);
	
	\pgfmathsetmacro{\h}{1}	
 	\pgfmathsetmacro{\dh}{1/6}
	
	% Trave
	\filldraw[fill=gray!5!white, line width=2pt, draw=black] 
	(a) -- (b) -- (c) -- (d) -- cycle;

	\begin{scope}[]
	\foreach \x in {1,2,...,5}
	{
	\draw[line width=0.7pt, draw=black] (\x*\dh,  0) -- (\x*\dh,\h);
	\draw[line width=0.7pt, draw=black] (0, \x*\dh) -- (\h, \x*\dh);
	}
	\end{scope}
		
\end{tikzpicture}
%\end{center}
%\caption{Square Problem (regular mesh)}
%\end{figure}
%
%\end{document}}
\hspace{5pt}
\subfigure[Trapezoidal mesh \label{fig:square_trapezoidal}]{%\documentclass{article}
%
%\usepackage{tikz}
%\usepackage{tikz-3dplot}
%\usetikzlibrary{calc}
%\usetikzlibrary{patterns}
%\usetikzlibrary{intersections}
%\usetikzlibrary{arrows}
%\tikzset{>=latex}
%
%\begin{document}
%
%\begin{figure}[!h]
%\begin{center}

\begin{tikzpicture}[scale = 5.]
	\coordinate[](a) at (0,0);
	\coordinate[](b) at (1,0);	
	\coordinate[](c) at (1,1);
	\coordinate[](d) at (0,1);
	
	\pgfmathsetmacro{\h}{1}	
 	\pgfmathsetmacro{\dh}{1/6}
	
	% Trave
	\filldraw[fill=gray!5!white, line width=2pt, draw=black] 
	(a) -- (b) -- (c) -- (d) -- cycle;

	\begin{scope}[]
	\foreach \x in {1,3,...,5}
	{
	\draw[line width=0.7pt, draw=black] 
	(0, \x*\dh-\dh/2) -- (\dh, \x*\dh+\dh/2) -- (2*\dh, \x*\dh-\dh/2) -- 
	(3*\dh, \x*\dh+\dh/2) -- (4*\dh, \x*\dh-\dh/2) -- (5*\dh, \x*\dh+\dh/2)
	-- (6*\dh, \x*\dh-\dh/2);
	}
	\end{scope}
	
	\begin{scope}[]
	\foreach \x in {1,2,...,5}
	{
	\draw[line width=0.7pt, draw=black] (\x*\dh,  0) -- (\x*\dh,\h);
	}
	\end{scope}
	
	\begin{scope}[]
	\foreach \x in {2,4,...,5}
	{
	\draw[line width=0.7pt, draw=black] (0, \x*\dh) -- (\h, \x*\dh);
	}
	\end{scope}	
	
	
		
\end{tikzpicture}
%\end{center}
%\caption{Square Problem (irregular mesh)}
%\end{figure}
%
%\end{document}}
\caption{Square Problem}
\end{center}
\end{figure}
%
\begin{figure}[h!]
\begin{center}
\subfigure[Regular mesh \label{fig:square_regular}]
{%\documentclass{article}
%
%\usepackage{graphicx}
%\usepackage{xcolor}
%\usepackage{colortbl}
%\usepackage{fp}
%\usepackage{tikz, pgfplots}
%\usetikzlibrary{calc}
%\usetikzlibrary{patterns}
%\usetikzlibrary{intersections}
%\usetikzlibrary{arrows}
%\tikzset{>=latex}
%
%% Color definition
%\definecolor{green2}{RGB}{154, 205, 50}
%\definecolor{green3}{RGB}{141, 182, 0}
%\definecolor{blue2}{RGB}{0, 102, 255}
%\definecolor{lightgreen}{RGB}{178,255,102}
%
%
%\begin{document}
%\scriptsize
%\begin{figure}[!h]
%\begin{center}
\begin{tikzpicture}
 %transpose legend
 \begin{loglogaxis}[width=0.45\textwidth, height=0.45\textwidth,
  legend style={anchor=south, at={(0.5, 1.07)}, draw=none, font=\scriptsize},
  grid=major,
  legend columns=2,
  xlabel={\small Number of Elements},
  ylabel={\small $\parallel \bm{u}_{h}-\bm{u} \parallel_{L^{2}} / \parallel \bm{u} \parallel_{L^{2}}$},
  xmin=1, xmax=1e4,
  ymin=1e-4, ymax=1e0,
  ytick={1e-4, 1e-3, 1e-2, 1e-1, 1},
  yticklabels={$10^{-4}$, $10^{-3}$, $10^{-2}$, $10^{-1}$, $10^{0}$},
  xtick={1e0, 1e1, 1e2, 1e3, 1e4},
  xticklabels={$10^{0}$, $10^{1}$, $10^{2}$, $10^{3}$, $10^{4}$},
 ]
 % Reference
 %\addplot[black, ultra thick] coordinates{
 %(50, 1.8248e-5)
 %(15000, 1.8248e-5)
 %}; 
 %\addlegendentry{Reference} 
 %
 \addplot[black, mark=square, thick, mark options={solid}]
 table [x index={0}, y index={1}]
 {figure/square/elastic_error_disp_u_peersq.txt};
 \addlegendentry{PEERSQ}
 % 
 \addplot[green, mark=triangle, thick, mark options={solid}]
 table [x index={0}, y index={1}]
 {figure/square/elastic_error_disp_u_peers_1b.txt};
 \addlegendentry{PEERSQ1B} 
 % 
 \addplot[red, mark=x, thick, mark options={solid}]
 table [x index={0}, y index={1}]
 {figure/square/elastic_error_disp_u_peers_2b.txt};
 \addlegendentry{PEERSQ2B}
 %
 \addplot[red, mark=o, thick, mark options={solid}]
 table [x index={0}, y index={1}]
 {figure/square/elastic_error_disp_u_abf_2b.txt};
 \addlegendentry{ABFQ2B}
 %
 \addplot[blue, mark=x, thick, mark options={solid}]
 table [x index={0}, y index={1}]
 {figure/square/elastic_error_disp_u_peers_2bmixed.txt};
 \addlegendentry{PEERSQ2BM }
 % 
 \addplot[blue, mark=o, thick, mark options={solid}]
 table [x index={0}, y index={1}]
 {figure/square/elastic_error_disp_u_abf_2bmixed.txt};
 \addlegendentry{ABFQ2BM }
 %
 %\addplot[red, mark=+, dashed, very thick]
 %table [x index={0}, y index={1}]
 %{};
 %\addlegendentry{}
 %
 %\addplot[blue, mark=x, very thick]
 %table [x index={0}, y index={1}]
 %{};
 %\addlegendentry{}
 %
 %\addplot[green3, mark=x, very thick]
 %table [x index={0}, y index={1}]
 %{};
 %\addlegendentry{}
 % 
 \end{loglogaxis}
\end{tikzpicture}
%\end{center}
%\caption{The relative error versus the number of elements measured relative 
%to the $L^{2}$}
%\end{figure}
%
%\end{document}
}	
\subfigure[Trapezoidal mesh \label{fig:square_trapezoidal}]{%\documentclass{article}
%
%\usepackage{graphicx}
%\usepackage{xcolor}
%\usepackage{colortbl}
%\usepackage{fp}
%\usepackage{tikz, pgfplots}
%\usetikzlibrary{calc}
%\usetikzlibrary{patterns}
%\usetikzlibrary{intersections}
%\usetikzlibrary{arrows}
%\tikzset{>=latex}
%
%% Color definition
%\definecolor{green2}{RGB}{154, 205, 50}
%\definecolor{green3}{RGB}{141, 182, 0}
%\definecolor{blue2}{RGB}{0, 102, 255}
%\definecolor{lightgreen}{RGB}{178,255,102}
%
%
%\begin{document}
%\scriptsize
%\begin{figure}[!h]
%\begin{center}
\begin{tikzpicture}
 %transpose legend
 \begin{loglogaxis}[width=0.45\textwidth, height=0.45\textwidth,
  legend style={anchor=south, at={(0.5, 1.07)}, draw=none, font=\scriptsize},
  grid=major,
  legend columns=2,
  xlabel={\small Number of Elements},
  ylabel={\small $\parallel u_{h}-u \parallel_{L^{2}}/ \parallel u \parallel_{L^{2}}$},
  xmin=1e0, xmax=1e5,
  ymin=1e-3, ymax=1e2,
  ytick={1e-3, 1e-2, 1e-1, 1, 1e1, 1e2},
  yticklabels={$10^{-3}$, $10^{-2}$, $10^{-1}$, $10^{0}$, 
  $10^{1}$, $10^{2}$},
  xtick={1e0, 1e1, 1e2, 1e3, 1e4, 1e5},
  xticklabels={$10^{0}$, $10^{1}$, $10^{2}$, $10^{3}$, $10^{4}$, $10^{5}$},
 ]
 % Reference
 %\addplot[black, ultra thick] coordinates{
 %(50, 1.8248e-5)
 %(15000, 1.8248e-5)
 %}; 
 %\addlegendentry{Reference} 
 %
 \addplot[green, mark=square, very thick, mark options={solid}]
 table [x index={0}, y index={1}]
 {figure/square/new_example/results/elastic_error_disp_u_peers_trapezoidal_1b.txt};
 \addlegendentry{PEERSQ1B}
 %
 \addplot[red, mark=x, very thick, mark options={solid}]
 table [x index={0}, y index={1}]
 {figure/square/new_example/results/elastic_error_disp_u_peers_trapezoidal_2b.txt};
 \addlegendentry{PEERSQ2B}
 %
 \addplot[blue, mark=o, very thick, mark options={solid}]
 table [x index={0}, y index={1}]
 {figure/square/new_example/results/elastic_error_disp_u_peers_trapezoidal_2bm.txt};
 \addlegendentry{PEERSQ2BM}
 %
 \addplot[black, mark=square, very thick]
 table [x index={0}, y index={1}]
 {figure/square/new_example/results/elastic_error_disp_u_peersq_trapezoidal.txt};
 \addlegendentry{PEERSQ}
 %
 %\addplot[blue, mark=x, very thick]
 %table [x index={0}, y index={1}]
 %{};
 %\addlegendentry{}
 %
 %\addplot[green3, mark=x, very thick]
 %table [x index={0}, y index={1}]
 %{};
 %\addlegendentry{}
 % 
 \end{loglogaxis}
\end{tikzpicture}
%\end{center}
%\caption{Trapezoidal mesh: relative error versus the number of elements measured relative 
%to the $L^{2}$}
%\end{figure}
%
%\end{document}
}
\caption{Error in norm-$L^{2}$ of square problem}
\end{center}
\end{figure}


\subsection{Cantilever beam problem}

\begin{figure}[!h]
\begin{center}
%\documentclass{article}
%
%\usepackage{tikz}
%\usepackage{tikz-3dplot}
%\usetikzlibrary{calc}
%\usetikzlibrary{patterns}
%\usetikzlibrary{intersections}
%\usetikzlibrary{arrows}
%\tikzset{>=latex}
%
%\begin{document}
%
%\begin{figure}[!h]
%\begin{center}
\Large
\begin{tikzpicture}[scale = 1]
	
	\pgfmathsetmacro{\L}{10}	
 	\pgfmathsetmacro{\l}{2}	
	\pgfmathsetmacro{\dl}{2}
	
	
	\coordinate[](a) at (0,0);
	\coordinate[](b) at (\L,0);	
	\coordinate[](c) at (\L,\l);
	\coordinate[](d) at (0,\l);
	
	\coordinate[](m) at (0,\l/2);
	\coordinate[](mm) at (\L,\l/2);
	
	\coordinate[](t) at (\L+\dl,\l);
	\coordinate[](tt) at (\L+\dl-1,\l);
	\coordinate[](u) at (\L+\dl,0);
	\coordinate[](uu) at (\L+\dl+1,0);
	
	%% Quote
	\coordinate[](ql) at (0, -\dl/2);
	\coordinate[](qll) at (\L, -\dl/2);
	\coordinate[](qh) at (-\dl, 0);
	\coordinate[](qhh) at (-\dl, \l);
		

	% Trave
	\filldraw[fill=gray!10!white, line width=2pt, draw=black] 
	(a) -- (b) -- (c) -- (d) -- cycle;
	
	\begin{scope}[]
	\foreach \x in {1,2,...,4}
	{
	\draw[line width=0.7pt, draw=black] (\x*\dl,  0) -- (\x*\dl,\l);
	}
	\end{scope}	
	
	% Vincoli
	\begin{scope}[]
	\foreach \x in {0,1,...,2}
	{
	\filldraw[fill=black, line width=0.7pt, draw=black] 
	(-0.25,\l-\x*0.65) circle (0.25);
	}
	\end{scope}	
	
	\draw[line width=2pt, draw=black] 
	(a) -- (-0.5,0.3) -- (-0.5,-0.3) -- cycle;
	
	 %% Load
	\begin{scope}[->, ultra thick]
	\draw[->,line width=2.0pt,red] (t) -- (tt) node[red,midway,above]
	{$\textbf{-f}$};
	\draw[->,line width=2.0pt,red] (u) -- (uu) node[red,midway,below]
	{$\textbf{f}$};
	\draw[->,line width=2.0pt,red] (\L+\dl,3*\dl/4) -- (\L+\dl-0.5,3*\dl/4);
	\draw[->,line width=2.0pt,red] (\L+\dl,\dl/4) -- (\L+\dl+0.5,\dl/4);
	\end{scope}	
	\draw[line width=0.7pt, black, dashed] (t) -- (u) ;
	\draw[line width=0.7pt, black, dashed] (tt) -- (uu) ;	
	
	\draw[line width=0.7pt, black, dashdotted] (-1,1) -- (13,1) ;
	%% Quote
	\draw [<->,color=black] (ql) -- (qll) node[black,midway,above] {$L$};
	\draw [<->,color=black] (qh) -- (qhh) node[black,midway,left] {$l$};
	
	% Tratteggio
	\fill[pattern=north west lines, pattern color=black] 
	(-0.75,-0.4) rectangle (-0.5,\l+0.25);
	
	 %% Axis
	\begin{scope}[->, thick]
	\draw[->,line width=1.0pt,black] (0,\l+0.5) -- (0,\l+\dl);
	\draw[->,line width=1.0pt,black] (0,\l+0.5) -- (\l-0.5,\l+0.5);
	\end{scope}	
	\node[black,left] at (0,\l+\dl) {$\textbf{y}$};
	\node[black,above] at (\l-0.5,\l+0.5) {$\textbf{x}$};	
		
\end{tikzpicture}
%\end{center}
%\caption{Beam Cantilever}
%\end{figure}
%
%\end{document}
\caption{Cantilever Beam: Geometry problems \label{fig:beam}}
\end{center}
\end{figure}


\subsection{Cook's membrane}

\begin{figure}[!h]
\begin{center}
%\documentclass{article}
%
%\usepackage{tikz}
%\usepackage{tikz-3dplot}
%\usetikzlibrary{calc}
%\usetikzlibrary{patterns}
%\usetikzlibrary{intersections}
%\usetikzlibrary{arrows}
%\tikzset{>=latex}
%
%\begin{document}
%
%\begin{figure}[!h]
%\begin{center}
\Large
\begin{tikzpicture}[scale = 0.1]
	\coordinate[](a) at (0,0);
	\coordinate[](b) at (48,44);	
	\coordinate[](c) at (48,60);
	\coordinate[](d) at (0,44);
	
	\coordinate[](Mu) at (50,44);
	\coordinate[](Muu) at (50,60);
	\coordinate[](Ml) at (50,52);
	
	\coordinate[](aa) at (0, 22);	
	\coordinate[](bb) at (48,52);
	\coordinate[](cc) at (24,22);
	\coordinate[](dd) at (24,52);
	
	\coordinate[](Ql) at (0,-5);	
	\coordinate[](Qll) at (48,-5);
	\coordinate[](Qh) at (-10,0);	
	\coordinate[](Qhh) at (-10,44);
	\coordinate[](Qhhh) at (-10,60);	
	
	% Trave
	\filldraw[fill=gray!10!white, line width=2pt, draw=black] (a) -- (b) -- (c) -- (d) -- cycle;
	\draw[line width=1pt, black] (aa) -- (bb);
	\draw[line width=1pt, black] (cc) -- (dd);
	
	% Tratteggio
	\fill[pattern=north west lines, pattern color=black] (d) rectangle (-3,0);
	
	 %% Load
	\begin{scope}[->, ultra thick]
	\draw[->,line width=2.0pt,red] (Mu) -- (Muu);
	\end{scope}	
	\node[red,right] at (Ml) {$\textbf{F}$};
		
	\fill[red] node at (c) {$\bullet$};	
	\node[red,above] at (c) {$\textbf{A}$};
	
	%% Quote
	\draw [<->,color=black] (Ql) -- (Qll) node[black,midway,above] {$48$};
	\draw [<->,color=black] (Qh) -- (Qhh) node[black,midway,left] {$44$};
	\draw [<->,color=black] (Qhh) -- (Qhhh) node[black,midway,left] {$16$};
			
\end{tikzpicture}
%\end{center}
%\caption{Cook's Membrane}
%\end{figure}
%
%\end{document}
\caption{Cook's Membrane Geometry \label{fig:cook_membrane}}
\end{center}
\end{figure}
%
\begin{figure}[!h]
\begin{center}
\subfigure[PEERS Element]{%\documentclass{article}

%\usepackage{graphicx}
%\usepackage{xcolor}
%\usepackage{colortbl}
%\usepackage{fp}
%\usepackage{tikz, pgfplots}
%\usetikzlibrary{calc}
%\usetikzlibrary{patterns}
%\usetikzlibrary{intersections}
%\usetikzlibrary{arrows}
%\tikzset{>=latex}
%
%% Color definition
%\definecolor{green2}{RGB}{154, 205, 50}
%\definecolor{green3}{RGB}{141, 182, 0}
%\definecolor{blue2}{RGB}{0, 102, 255}
%\definecolor{lightgreen}{RGB}{178,255,102}


%\begin{document}

%\begin{figure}[!h]
%\begin{center}
\begin{tikzpicture}
 %transpose legend
 \begin{axis}[width=0.45\textwidth, height=0.45\textwidth,
  legend style={anchor=south, at={(0.5, 1.07)}, draw=none, font=\scriptsize},
  legend columns=2,
  xlabel={\small Number of Elements per Side},
  ylabel={\small Vertical displacement of point A},
  xmin=0, xmax=66,
  ymin=10, ymax=16,
  ytick={5, 6, 7, 8, 9, 10, 11, 12, 13, 14, 15, 16},
  yticklabels={$5$, $6$, $7$, $8$, $9$, $10$, $11$, $12$, $13$, $14$, $15$, $16$},
  xtick={0, 4, 8, 16, 32, 64},
  xticklabels={$0$, $4$, $8$, $16$, $32$, $64$},
 ]
 % Reference
 %\addplot[black, ultra thick] coordinates{
 %(50, 1.8248e-5)
 %(15000, 1.8248e-5)
 %}; 
 %\addlegendentry{Reference} 
 %
 \addplot[black, mark=square, thick, mark options={solid}]
 table [x index={0}, y index={1}]
 {figure/cook/cook_regular_peersq.txt};
 \addlegendentry{PEERSQ} 
 %
 \addplot[green, mark=triangle, thick, mark options={solid}]
 table [x index={0}, y index={1}]
 {figure/cook/cook_regular_peers_1b_v1.txt};
 \addlegendentry{PEERSQ1B (v.1)}
 %
 \addplot[red, mark=o, thick, mark options={solid}]
 table [x index={0}, y index={1}]
 {figure/cook/cook_regular_peers_2b.txt};
 \addlegendentry{PEERSQ2B} 
 % 
 \addplot[blue, mark=x, thick, mark options={solid}]
 table [x index={0}, y index={1}]
 {figure/cook/cook_regular_peers_2bm.txt};
 \addlegendentry{PEERSQ2BM}


% %
% \addplot[green3, mark=o, thick, dashed, mark options={solid}]
% table [x index={0}, y index={1}]
% {figure/cook/cook_regular_peers_ver3_ver2.txt};
% \addlegendentry{PEERSQ1B (v.2)}
% % 

 \end{axis}
\end{tikzpicture}
%\end{center}
%\caption{Vertical Displacement of point A vs. the number of element per side}
%\end{figure}

%\end{document}}
\subfigure[ABF Element]{%\documentclass{article}

%\usepackage{graphicx}
%\usepackage{xcolor}
%\usepackage{colortbl}
%\usepackage{fp}
%\usepackage{tikz, pgfplots}
%\usetikzlibrary{calc}
%\usetikzlibrary{patterns}
%\usetikzlibrary{intersections}
%\usetikzlibrary{arrows}
%\tikzset{>=latex}
%
%% Color definition
%\definecolor{green2}{RGB}{154, 205, 50}
%\definecolor{green3}{RGB}{141, 182, 0}
%\definecolor{blue2}{RGB}{0, 102, 255}
%\definecolor{lightgreen}{RGB}{178,255,102}


%\begin{document}

%\begin{figure}[!h]
%\begin{center}
\begin{tikzpicture}
 %transpose legend
 \begin{axis}[width=0.45\textwidth, height=0.45\textwidth,
  legend style={anchor=south, at={(0.5, 1.07)}, draw=none, font=\scriptsize},
  legend columns=2,
  xlabel={Number of Elements per Side},
  ylabel={Vertical displacement of point A},
  xmin=0, xmax=32,
  ymin=5, ymax=9,
  ytick={5, 6, 7, 8, 9},
  yticklabels={$5$, $6$, $7$, $8$, $9$},
  xtick={0, 2, 4, 8, 16, 32},
  xticklabels={$0$, $2$, $4$, $8$, $16$, $32$},
 ]
 % Reference
 %\addplot[black, ultra thick] coordinates{
 %(50, 1.8248e-5)
 %(15000, 1.8248e-5)
 %}; 
 %\addlegendentry{Reference} 
 %
 \addplot[red, mark=+, thick, mark options={solid}]
 table [x index={0}, y index={1}]
 {figure/cook/cook_regular_abf_ver1.txt};
 \addlegendentry{ABF \textbf{2B} (mixed)}
 %
 \addplot[blue, mark=x, thick, mark options={solid}]
 table [x index={0}, y index={1}]
 {figure/cook/cook_regular_abf_ver2.txt};
 \addlegendentry{ABF \textbf{2B}}
 %
 \addplot[green3, mark=o, thick, mark options={solid}]
 table [x index={0}, y index={1}]
 {figure/cook/cook_regular_abf_ver3_ver1.txt};
 \addlegendentry{ABF \textbf{1B} (version 1)}
 %
 \addplot[green3, mark=o, thick, dashed, mark options={solid}]
 table [x index={0}, y index={1}]
 {figure/cook/cook_regular_abf_ver3_ver2.txt};
 \addlegendentry{ABF \textbf{1B} (version 2)}
 % 
 \end{axis}
\end{tikzpicture}
%\end{center}
%\caption{Vertical Displacement of point A vs. the number of element per side}
%\end{figure}

%\end{document}}
\caption{Cook's Membrane: Vertical Displacement of Point A vs. Element per Side}
\end{center}
\end{figure}


\end{document}